% !TEX TS-program = lualatex
\documentclass[12pt]{article}

\usepackage{luacode}
\begin{luacode}
function paren_array(k,l,m,n)
  for i = k,m do
    for j = l,n do
      tex.sprint("("..i..","..j..")  ")
    end
    tex.print("")  -- force a line break
  end
end

function subscript_array(ind,k,l,m,n)
  for i=k,m do
    for j=l,n do
      tex.sprint(ind.."_{"..i..j.."}")
      if j<n then
        tex.sprint ( "&" )    -- cell separator (ampersand)
      else
        tex.sprint( "\\\\" )  -- end of row (two backslashes)
      end
    end
  end
end
\end{luacode}

\usepackage[ownhref,ccsgyak,thmhu]{polcz}
\usepackage{amsmath,amssymb,mathtools}
\usepackage{xcolor}
\usepackage{lipsum}

\usepackage[
    hyperfootnotes = false,
    colorlinks = true,
    linkcolor = blue,
    urlcolor  = blue,
    citecolor = blue,
    anchorcolor = blue,
    pagebackref = false
    ]{hyperref}
\usepackage[utf8]{inputenc}

\makeatletter
\renewcommand*\env@matrix[1][*\c@MaxMatrixCols c]{%
  \hskip -\arraycolsep
  \let\@ifnextchar\new@ifnextchar
  \array{#1}}
\makeatother

\providecommand{\pfor}[2]{\left[ \foreach \n in {#1}{\n \middle|} #2 \right]}

\makeatletter
\DeclareDocumentCommand\Features{ s m O{0} O{0} m m }{%
  \global\let\MatrixBody\@empty
  \def\TMPvarname{#2}
  \foreach \entry/\index in {#5/100} {%
    \ifnum \index < 100
        \expandafter\g@addto@macro\expandafter\MatrixBody
        \expandafter{%
            \expandafter\TMPvarname\entry &
        }%
    \else
        \expandafter\g@addto@macro\expandafter\MatrixBody
        \expandafter{%
            \expandafter\TMPvarname\entry
        }%
    \fi
  }%
    \begin{bmatrix}%
        \MatrixBody
    \end{bmatrix}%
}
\makeatother

\begin{document}
    $$
        \innerp*{\frac12}{a}
    $$

    Loop: \\
    \foreach \x/\index in {4,5,4,2/0} {
        \x - \index
        \ifnum \index > 0
            -- kutya
        \fi
        \\
    }

    $
    \bmatrix
        a & b
    \endbmatrix
    $

    $\Features{a}{1,2,3}{}$

    \begin{equation}
        \begin{bmatrix}
            a & c
        \end{bmatrix}
        \begin{bmatrix}[c|c]
            a & c
        \end{bmatrix}
        ~~~~~~~~~~~~~
        \left[\mqty{\xmat*{a}{2}{2}} ~\middle|~ \mqty{\xmat*{a}{2}{2}} \right]
    \end{equation}

    \newcount\i
    \newcount\j

    \newcount\i
    \newcount\j
    \i = 2
    \loop
    \j = 2
    \advance \i by 1
    {
        \loop
        \advance \j by 1
        (\the\i,\the\j)
        \ifnum \j < 6 \repeat
    }
    \ifnum \i < 5
        \global\matrixtoks = \expandafter{\the\matrixtoks \\ }
    \repeat

    \i=0
    \loop
        % increment dummy counter
        \advance\i by 1

        % \j=0
        % \loop

        %     (\the\i,\the\j)

        %     \ifnum\j<6
        % \repeat

        % repeat the loop provided the counter is within specified bound
        \ifnum\i<5
    \repeat


    \begin{equation}
        \qty<1> \dqty{\frac{a}{b},b}
        \qty(a)[b]{c}
    \end{equation}
    \begin{equation}
        \lbrack \lvert
    \end{equation}

    $\foreach \n in {1,3}{\n,}$

    \langol[label]{Kutyagumi}

    \begin{pczLearnEnvironment}*\alpha\asdas[b]{}|c|<d>(required argument)<re>
    \end{pczLearnEnvironment}

    \begin{pczLearnEnvironment}\alpha{}(required argument)<re>
    \end{pczLearnEnvironment}

    \noindent
    \color{black}
    \TODO{Kutyagumi}
    \TODOx{Kutyagumi}

    \noindent
    Kutyagumi asdasdasdasd asd asdas dasd as asd as
    asdasda sd as asd asd asd asd asd ad asd as

    Kutyagumi(black)

\angol[Magyarazat]{Szo}
\angolI[Magyarazat]{Szo}

\refangol[label]{Kutyaurulek}

\listoftodosangol

\tcbset{enhanced}
\begin{tcolorbox}[breakable,title=My breakable box]
  \lipsum[1-6]
\end{tcolorbox}

\normalfont
\begin{cmatlab}[\tt kutya]{Gumi}
    % \lipsum[1]
    asdasd
    \tcbsubtitle{Subtitle \thepczmatlab.1}
    % \lipsum[1]
    asdasd
\end{cmatlab}


% \Normal\n
\section{LuaCode}
\newcommand{\luapmat}[4]{%
  $\directlua{ paren_array(\luatexluaescapestring{#1},\luatexluaescapestring{#2},\luatexluaescapestring{#3},\luatexluaescapestring{#4}) }$
}


\newcommand*{\mycommand}{%
  \directlua{for i = 1, 10 do tex.sprint(i .. "\luatexluaescapestring{\hskip1cm}") end}
}

\mycommand

%% call the first function
\directlua{ paren_array(3,4,6,6) }

\bigskip
%% call the second function from within a LaTeX "array" environment
$
    A = \left[ \begin{array}{*{5}{c}}
       \directlua{ subscript_array("a",2,3,5,5 ) }
    \end{array} \right]
$

$
    \luapmat{2}{3}{4}{4}
$

\newpage
\begin{align*}
    \hmath[remember as=fx]{f(x)}
    &= \int\limits_{1}^{x} \frac{1}{t^2}~dt = \left[ -\frac{1}{t} \right]_{1}^{x}\\
    &= -\frac{1}{x} + \frac{1}{1}\\
    &=
    \hmath[
        overlay={
            % \draw[blue,very thick,->] (fx.south) to[bend right] ([yshift=2mm]frame.west);
            \draw[bourdon,very thick,->] (fx.south) to[bend right] (frame.160);
        }
    ][white]{1-\frac{1}{x}.}
\end{align*}
text
\begin{align}
\hmath[
    overlay={
        % \draw[blue,very thick,->] (fx.south) to[bend right] ([yshift=2mm]frame.west);
        \draw[bourdon,very thick,->] (fx.south) to[bend right] (frame.140);
    }
]{1-\frac{1}{x}}
\end{align}

\begin{equation*}
\vec{a}_p = \vec{a}_o+\frac{{}^bd^2}{dt^2}\vec{r} +
        \tikz[baseline,remember picture]{
            \draw
                node [fill=red!20,anchor=base] (eq) {$\displaystyle2\vec{\omega}_{ib}\times\frac{{}^bd}{dt}\vec{r}$}
                node at (eq.south west) [coordinate] (t1) {}
                node at (eq.south east) [coordinate] (t2) {};
        }
        \tikz[remember picture,overlay]{\draw[<->] (t1) -- node[below] {$n$} (t2); }
        +
        \tikz[baseline]{
            \node[fill=red!20,anchor=base] (t2)
            {$\vec{\alpha}_{ib}\times\vec{r}$};
        }
        +
        \tikz[baseline]{
            \node[fill=green!20,anchor=base] (t3)
            {$\vec{\omega}_{ib}\times(\vec{\omega}_{ib}\times\vec{r})$};
        }
\end{equation*}

\begin{align}
    I = \frac{1}{1-x} \int_0^1 \frac{1}{\sqrt{1 + x^2}} \diffd x
\end{align}

\newtcbtheorem[number within=section]{mytheo}{My Theorem}%
{colback=green!5,colframe=green!35!black,fonttitle=\bfseries}{th}

\begin{mytheo}{This is my title}{theoexample}
This is the
 text of the theorem. The counter is automatically assigned and,
in this example, prefixed with the section number. This theorem is numbered with
\ref{th:theoexample} and is given on page \pageref{th:theoexample}.
\end{mytheo}

Lásd: \ref{pl:marker}

\begin{pcb}[title=Kutyagumi][blue]
    This is my own box with a mandatory title and options.
\end{pcb}

\begin{pcb}[title=Kutyagumi,oversize][bourdon]<bourdon!35!yellow>
    This is my own box with a mandatory title and options.
\end{pcb}
\end{document}